\documentclass[10pt,handout,english]{beamer}
\usetheme{Warsaw}
\usepackage{graphicx}
\usepackage{placeins}
\usepackage{amsmath}
\usepackage{tabu}
\usepackage{bbm}
\usepackage{booktabs}
\usepackage[round]{natbib}
\usepackage{bm}
\usepackage{ragged2e}
\usepackage{hyperref}
\usepackage{amsmath}
\usepackage{xcolor}
\usepackage[super]{nth}
\hypersetup{
    colorlinks=true,
    linkcolor=blue,
    filecolor=blue,      
    urlcolor=blue,
    citecolor=black,
}

\apptocmd{\frame}{}{\justifying}{} % Allow optional arguments after frame.

\setbeamertemplate{frametitle continuation}{}

\newcommand\setItemnumber[1]{\setcounter{enumi}{\numexpr#1-1\relax}}

\newcommand{\ts}{\textsuperscript}
\newcommand{\E}{\mathbb{E}}
\newcommand{\R}{\mathbb{R}}
\newcommand{\F}{\mathcal{F}}

\title[]{Random matrices and covariance estimation}
\author[Kaveh S. Nobari]{Kaveh S. Nobari}
\institute[]{Lectures in High-Dimensional Statistics}
\date[27/10/2020]
{Department of Mathematics and Statistics\\ Lancaster University}
	

\begin{document}
\begin{frame}
\titlepage
\end{frame}


\begin{frame}{Contents}
\tableofcontents
\end{frame}

\begin{frame}[allowframebreaks]{Motivation}
The issue of covariance estimation is intertwined with random matrix theory, since sample covariance is a particular type of random matrix. These slides follow the structure of chapter 6 of \citet{wainwright2019high} to shed light on random matrices in a \textcolor{red}{non-asymptotic setting}, with the aim of \textcolor{red}{obtaining explicit deviation inequalities that hold for all sample sizes and matrix dimensions.}
{\vskip 0.5em}
In the classical framework of covariance matrix estimation the sample size $n$ tends to infinity while the matrix dimension $d$ is fixed; in this setting the behaviour of sample covariance matrix is characterized by the usual limit theory. In contrast, in high-dimensional settings the data dimension is either comparable to the sample size $(d\asymp n)$ or possibly much larger than the sample size $d\gg n$.
{\vskip0.5em}
We begin with the simplest case, namely ensembles of Gaussian random matrices, and we then discuss more general sub-Gaussian ensembles, before moving to milder tail conditions.  
\end{frame}

\section{Preliminaries}
\frame{\tableofcontents[currentsection]}


%------------------------------------------------
\begin{frame}[allowframebreaks]
Consider a rectangular matrix $A\in\R^{n\times m}$ with $n\geq m$, the ordered singular values are written as follows
\[
\sigma_{\max}(A)=\sigma_1(A)\geq\sigma_{2}(A)\geq\cdots\geq\sigma_m(A)=\sigma_{\min}(A)\geq 0
\]

\end{frame}

\section{Wishart matrices and their behaviour}
\frame{\tableofcontents[currentsection]}


\section{Covariance matrices from sub-Gaussian ensembles}
\frame{\tableofcontents[currentsection]}

\section{Bounds for general matrices}
\subsection{Background on matrix analysis}
\frame{\tableofcontents[currentsection,currentsubsection]}


\subsection{Tail conditions for matrices}
\frame{\tableofcontents[currentsection,currentsubsection]}


\subsection{Matrix Chernoff approach and independent decompositions}
\frame{\tableofcontents[currentsection,currentsubsection]}


\subsection{Upper tail bounds for random matrices}
\frame{\tableofcontents[currentsection,currentsubsection]}


\subsection{Consequences for covariance matrices}
\frame{\tableofcontents[currentsection,currentsubsection]}

\section{Bounds for structured covariance matrices}
\subsection{Unknown sparsity and thresholding}
\frame{\tableofcontents[currentsection,currentsubsection]}

\begin{frame}[allowframebreaks]
\frametitle{References}
\bibliographystyle{apa}
\bibliography{References_HDStat}
\end{frame}

\end{document}